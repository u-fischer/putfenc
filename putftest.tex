\documentclass[parskip=full-]{scrartcl}
\usepackage{graphicx}
\usepackage[PUTF, %lädt die Definitionen für den qrcode
            LGR,  % wird benötigt für die utf8-Eingabe von αβδ
            T1    % die normale Textkodierung
            ]{fontenc}

\usepackage{textcomp} %wird benötigt für die utf8-Eingabe von € und ¼½

\usepackage{xpatch}
\usepackage[forget]{qrcode} %forget damit qrcode nicht versucht die Sachen zu speichern

\makeatletter
%qrcode wird gepatcht, damit es zur anderen Kodierung wechselt:
\xpatchcmd\qrcode@i{\begingroup}{\begingroup\fontencoding{PUTF}\fontfamily{pdf}\selectfont}{}{\show\failed}

% Unterdrücken der aux-Ausgabe:
\def\qr@writebinarymatrixtoauxfile#1{}%
\makeatother

%Ein proforma Schrift.
\DeclareFontFamily{PUTF}{pdf}{}%
\DeclareFontShape{PUTF}{pdf}{m}{n}{ <-> ecrm1000 }{}%
\DeclareFontSubstitution{PUTF}{pdf}{m}{n}%

\DeclareUnicodeCharacter{2740}{Blume}
\begin{document}
\section{QR Codes mit Umlauten}

 Das Argument von \verb+\qrcode+ kann nun Umlaute und auch Symbole wie Emoji enthalten.

 Direkte utf8-Eingabe \fbox{❀} ist möglich, aber nur wenn die utf8-Zeichen auch außerhalb des \verb+\qrcode+-Befehls nutzbar sind, ohne  eine derartige Fehlermeldung zu erzeugen:

 \begin{verbatim}
 ! Package inputenc Error: Unicode char ❀ (U+2740)
 (inputenc)                not set up for use with LaTeX.
 \end{verbatim}

 Ggfs. muss man mit \verb+DeclareUnicodeCharacter+ (oder dem newunicodechar-Paket) eine Deklaration hinzufügen oder über Pakete (textcomp) oder Kodierungen (LGR) laden.
 Die meisten Befehle wie \verb+\textPlane+ usw sind außerhalb des Arguments von \verb+\qrcode+ nicht nutzbar.


\qrcode[nolink]{Grüße € é § ¼½αβδ
                 # % & $  ^ _ ~
                 \textFiveFlowerPetal ❀ \textPlane
                 \textinfty\textint
                 \textEmojiDuck
                 \textManFace \textLadiesroom \textEmojiSmileSunglass \textEmojiCalendar}%



 \bigskip

 Steht \verb+\qrcode+ in einem Argument, müssen \TeX-Sonderzeichen maskiert werden. Neue Zeilen kann man mit \verb+\textLF+ einbauen:

\fbox{\qrcode[nolink]{Grüße € é § ¼½αβδ\textLF
                 \# \% \& \$ \^ \_ \~\textLF
                \textFiveFlowerPetal \textPlane\textLF
                \textinfty\textint\textLF
                \textManFace \textLadiesroom \textEmojiSmileSunglass \textEmojiCalendar
                \textEmojiDuck
                 }}%


putfenc.def enthält \emph{nicht} für alle Unicodezeichen Definitionen. Insgesamt sind dort knapp 2000 enthalten. Es können aber problemlos weitere hinzugefügt werden.


\end{document} 